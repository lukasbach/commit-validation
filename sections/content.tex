\section{Introduction}

\subsection{Motivation}
\begin{itemize}
	\item Motivate Software engineering and QA during software engineering
	\item Motivate Commit Validation as method to reduce engineering costs
	\item Shortly mention some Commit Validation approaches as examples (CLEVER \cite{Nayrolles2018}, CommitGuru \cite{Rosen2015}, maybe Kamei? \cite{Kamei2013})
\end{itemize}

\subsection{Problem Definition}
\begin{itemize}
	\item Problem Definition of this work
\end{itemize}

\subsection{Scope of This Paper}
\label{sec:scope}
\begin{itemize}
	\item Which kinds of work have been considered for this paper?
	\item Why have naive commit-checking techniques such as just "tests are green, coverage is high enough, sonarqube gates passed" not been considered?
	\item Why have works based on fault-detection unrelated to commits not been considered for this paper?
\end{itemize}

\subsection{Outline}
\begin{itemize}
	\item Outline of this work
\end{itemize}


\section{Background on Commit Validation}

\subsection{Commit Validation Process}
\label{sec:cvprocess}
\begin{itemize}
	\item Basics on Commit-based QA
	\item How is Commit Validation implemented in a developers workflow?
\end{itemize}

\subsection{Just-In-Time Fault Detection}
(TODO: Some of these subsections might have to be moved to section \ref{sec:cvprocess})
\begin{itemize}
	\item Why is this necessary? How does this benefit developers?
	\item How does a potentially "faulty" commit look like? (unusually big, commits at unusual times, touching rarely touched files, see \cite{Goyal2017})
	\item Describe how bug-tracking/issue-tracking systems can be used to gather data about bug-introducing commits and matching bugfix-commits
	\item Describe how this can be implemented using static metrics extracted from commits
	\item Describe how this can be implemented using pattern recognition on commits
\end{itemize}

\subsection{Just-In-Time Fault Prevention}
(TODO: maybe not the best title. What I mean here is automatic patch generation as described in \cite{Nayrolles2018}. Maybe just use the title "Automatic Patch Generation"?)
\begin{itemize}
	\item Motivate why this benefits developers over just manually fixing detected bugs.
	\item Describe idea of how to implement this using pattern recognition on commits
\end{itemize}


%\subsection{Technical Approaches}
%\cite{Goyal2017,Rosen2015,Nayrolles2018,Kamei2013,Yang2015}


\section{Comparing Commit Validation Approaches}

\subsection{Classification Scheme for Commit Validation Approaches}
\label{sec:scheme}
\begin{itemize}
	\item How do relevant approaches differ from each other?
	\item List of criteria derived to compare Commit Validation approaches. For each criterium (roughly in one paragraph):
	\begin{itemize}
		\item Explanation for the criterium, what it means
		\item How is the criterium measured/derived from an Commit Validation approach
		\item Why is this criterium important for comparing Commit Validation approaches?
		\item For which use case context is this criterium relevant?
	\end{itemize}
	
\end{itemize}

\subsection{Search Process}
\begin{itemize}
	\item Describe why CLEVER was used as starting position for research for this paper. \cite{Nayrolles2018}
	\item Describe how other approaches have been found based on CLEVER and how they match the papers scope as described in \ref{sec:scope}.
\end{itemize}

\subsection{Threads to Validity}
\begin{itemize}
	\item Could there be any biases when choosing and comparing Commit Validation approaches? (See \cite{Kitchenham2004})
	\item How did this paper make sure not to be biased?
\end{itemize}


\section{Comparison of Commit Validation Approaches}

\subsection{Comparison Results}
\begin{itemize}
	\item Table mapping all metrics/criteria (described in \ref{sec:scheme}) and all 5 approaches to results
\end{itemize}

(TODO: Subsections for each categories of approaches follow. I'm not entirely sure yet how the different approaches can be distributed on the categories, if the categories are well defined enough and how they should be ordered in this paper. The following is an initial suggestion for categories.)
\subsection{Results for Metric Based Approaches}
\begin{itemize}
	\item Description for the results for metric based approaches such as Commit-Guru (\cite{Nayrolles2018}) and Kamei's approach (\cite{Kamei2013}).
\end{itemize}
\subsection{Results for Approaches Based on Anomaly Detection}
(TODO: I'm not sure yet if anomaly-based approaches differ from metric-based approaches enough to justify an additional subsection)
\begin{itemize}
\item Description for the results for approaches based on anomaly detection such as UnusualCommit (\cite{Goyal2017}).
\end{itemize}
\subsection{Results for Machine Learning Approaches}
\begin{itemize}
\item Description for the results for machine learning approaches such as Deeper (\cite{Yang2015}) or CLEVER (\cite{Nayrolles2018}, could also be a metric-based approach).
\end{itemize}



\section{Discussion of Comparison Results}
\begin{itemize}
	\item Interesting findings from the results?
	\item How can categories of Commit Validation techniques be mapped to use cases?
	\item How can this mapping be used to find an suitable Commit Validation technique for a new project?
\end{itemize}


\section{Related Surveys on Commit Validation}
\begin{itemize}
	\item Reference and description on other surveys such as \cite{Kim2008,Catolino2019,Syed2019,Yang2016}
	\item For each survey: How does its scope differ from the scope of this paper?
\end{itemize}


\section{Conclusions}

\subsection{Summary}
\begin{itemize}
	\item Short summary on the comparison scheme, which relevant criteria have been defined.
	\item Which interesting findings could be derived from the comparison results?
\end{itemize}

\subsection{Contributions}
\begin{itemize}
	\item How does the proposed comparison scheme contribute to the scientific topic of Commit Validation?
	\item Why are the results and the findings from them relevant for the topic?
\end{itemize}

\subsection{Future Work}
\begin{itemize}
	\item Which relevant parts have not been covered by this work? Why are they still relevant for the topic?
\end{itemize}
