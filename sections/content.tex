\section{Introduction}

\subsection{Motivation}
\begin{itemize}
	\item Motivate Software engineering and QA during software engineering.
	\item Motivate Commit Validation as method to reduce engineering costs.
	\item Shortly mention some Commit Validation approaches as examples (CLEVER \cite{Nayrolles2018}, CommitGuru \cite{Rosen2015}, maybe Kamei? \cite{Kamei2013}).
\end{itemize}

\subsection{Problem Definition}
%\begin{itemize}
%	\item Discuss the Problem Definition of this work
%	\begin{itemize}
%		\item Describe the problems that motivate existing approaches.
%		\item Describe the goals that this paper tries to fulfill and discuss how it realizes a solution for these goals.
%		\item State the success criteria for this work.
%	\end{itemize}
%\end{itemize}

[TODO should the background be defined here or in the motivation?]

There exists different tools with varying techniques on the topic of Commit Validation. As this is an emerging topic with many published works in the past few years, a clear State of the Art approach has not been defined yet, and it is hard to choose a suitable Commit Validation technique for a new project to leverage its benefits. 

The goal of this paper is to explore how Commit Validation techniques can be compared in an objective way, and to define a method of how to choose a fitting Commit Validation technique for a Software Engineering project.

To satisfy this goal the paper will propose an objective evaluation schema to compare existing Commit Validation techniques in an objective and fair way. Then a selection of five recent relevant technical approaches will be compared using the proposed evaluation schema in an effort to give guidelines for determining which approaches are suitable in which context. [TODO reference to where in this paper these approaches are listed/specified]

The following success criteria have been defined for this paper: 

\begin{itemize}
	\item The proposed comparison schema does not take any considerations into account that are not relevant for Commit Validation approaches.
	\item The proposed comparison schema takes the context for which the compared approaches were designed for into account.
	\item The comparison result of the compared approaches is specified in an explanatory way that helps potential readers to see for which use case the approach is suitable for.
	\item The paper serves as guidelines for readers to find a suitable Commit Validation technique for their use cases.
\end{itemize}

%  When working with commit based Version Management Systems, manual effort and costs can be significantly reduced by leveraging methods of automated Fault Prediction and Fault Prevention. As this is an emerging topic with many published works in the past few years, a clear State of the Art approach has not been defined yet. The goal of this work is to introduce the reader to the concept of commit validation, introduce 5 relevant approaches that have been implemented and compare them with an evaluation scheme that will be proposed. The evaluation scheme should compare the approaches in fair way to give insights to their effectiveness and use cases.


\subsection{Scope of This Paper}
\label{sec:scope}
\begin{itemize}
	\item Describe which kinds of work have been considered for this paper.
	\item Describe why naive commit-checking techniques such as just "tests are green, coverage is high enough, sonarqube gates passed" have not been considered.
	\item Discuss why works based on fault-detection unrelated to commits have not been considered for this. paper.
\end{itemize}

\subsection{Outline}
\begin{itemize}
	\item Describe the outline of this work.
\end{itemize}


\section{Background on Commit Validation}

This chapter gives an introduction on the topic of Commit Validation. First the process of Commit Validation, its target and how it is implemented in a developers workflow is described, then its two major components, Just-in-Time Fault-Detection and Just-in-Time Fault Prevention are specified. 

\subsection{Commit Validation Process}
\label{sec:cvprocess}

%\begin{itemize}
%	\item Introduce the basics on Commit-based QA.
%	\item Outline how Commit Validation is implemented in a developers workflow.
%\end{itemize}

%TODO in a previous chapter, define "Commit Validation" and other names that are used in literature, and maybe that this name is specifically used for this paper

There are many ways of increasing software quality in the field of software engineering. While the field is very broad, there are many tools and technical utilities that have been established as part of a state-of-the-art technology stack. Among others, that also includes \define{version-control systems}{VCS}. A version-control system is used to track the evolution of code projects and enable collaborative teams of software developers to cooperate on a consistent code base. The currently most used VCS is \textit{Git} (TODO cite). \cite{Chacon:2014:PG:2695634}

An important concept for version-control systems are \textit{commits}, small sets of code changes that usually happen atomically and are annotated by the developer describing what the changes do. Commits are explicitly performed by the developer and are usually mark a finished feature, bug-fix, chore work or similar artifacts. Because of that, the time where a developer performs a commit is a suitable time for validating the change and analyzing it to find potential bugs that have been introduced by the change while the developer still has the changes in his mind, yet considers them to be final. %TODO citation?

The VCS Git supports a concept named \textit{Hooks}. A hook specifies a custom script which runs programatically in response to a event such as commits or uploading a set of commits to a remote server (which is called a \textit{Push}-event in Git). Such hooks are differentiated into \textit{Client-Side Hooks}, which run on the local device of the developer that is authoring the commit, and \textit{Server-Side Hooks}, which run on the remote server. \cite{Chacon:2014:PG:2695634}

There exists a variety of approaches for analyzing the quality of code changes at commit time. A popular approach is the manual authoring of unit-, integration- or end-to-end-tests to verify that a project's implementation fulfills its specification \cite{Maayan2018}. Such test definitions can be setup as a commit hook for them to run at commit time.

However, this paper focuses on approaches which do not require manual specifications such as test cases to detect a faulty commit, but instead automatically rate the probability of a commit introducing a bug by leveraging external data sources that are available.

\subsection{Just-In-Time Fault Detection}
%TODO (TODO: Some of these subsections might have to be moved to section \ref{sec:cvprocess})
\begin{itemize}
	\item Introduce Just-in-Time Fault Detection.
	\item Discuss why Just-in-Time Fault Detection is necessary and how it benefits developers.
	\item Showcase how a potentially "faulty" commit looks like (unusually big, commits at unusual times, touching rarely touched files, see \cite{Goyal2017}).
	\item Describe how bug-tracking/issue-tracking systems can be used to gather data about bug-introducing commits and matching bugfix-commits.
	\item Describe how this can be implemented using static metrics extracted from commits.
	\item Describe how this can be implemented using pattern recognition on commits.
\end{itemize}

\subsection{Just-In-Time Fault Prevention}
%TODO (TODO: maybe not the best title. What I mean here is automatic patch generation as described in \cite{Nayrolles2018}. Maybe just use the title "Automatic Patch Generation"?)
\begin{itemize}
	\item Introduce Just-in-Time Fault Prevention and distinguish from Just-in-Time Fault Detection.
	\item Motivate why this benefits developers over just manually fixing detected bugs.
	\item Describe idea of how to implement this using pattern recognition on commits.
\end{itemize}


%\subsection{Technical Approaches}
%\cite{Goyal2017,Rosen2015,Nayrolles2018,Kamei2013,Yang2015}


\section{Comparing Commit Validation Approaches}

\subsection{Classification Scheme for Commit Validation Approaches}
\label{sec:scheme}
\begin{itemize}
	\item Describe how relevant approaches differ from each other.
	\item Derive list of criteria to compare Commit Validation approaches. For each criterium (roughly in one paragraph):
	\begin{itemize}
		\item Explain the criterium and what it means.
		\item Discuss how the criterium is measured and derived from an Commit Validation approach.
		\item Highlight the importance of this criterium in respect to comparing Commit Validation approaches.
		\item Describe use cases for which this criterium is relevant.
	\end{itemize}
	
\end{itemize}

\subsection{Search Process}
\begin{itemize}
	\item Describe why CLEVER was used as starting point for research for this paper. \cite{Nayrolles2018}
	\item Describe how other approaches have been found based on CLEVER and how they match the papers scope as described in \ref{sec:scope}.
\end{itemize}

\subsection{Threads to Validity}
\begin{itemize}
	\item Discuss if there can be any biases when choosing and comparing Commit Validation approaches (See \cite{Kitchenham2004}).
	\item Describe how this paper made sure not to be biased.
\end{itemize}


\section{Comparison of Commit Validation Approaches}

\subsection{Comparison Results}
\begin{itemize}
	\item Table mapping all metrics/criteria (described in \ref{sec:scheme}) and all 5 approaches to results.
\end{itemize}

%TODO (TODO: Subsections for each categories of approaches follow. I'm not entirely sure yet how the different approaches can be distributed on the categories, if the categories are well defined enough and how they should be ordered in this paper. The following is an initial suggestion for categories.)
\subsection{Results for Metric Based Approaches}
\begin{itemize}
	\item Describe the results for metric based approaches such as Commit-Guru (\cite{Nayrolles2018}) and Kamei's approach (\cite{Kamei2013}).
\end{itemize}
\subsection{Results for Approaches Based on Anomaly Detection}
%TODO (TODO: I'm not sure yet if anomaly-based approaches differ from metric-based approaches enough to justify an additional subsection)
\begin{itemize}
\item Describe the results for approaches based on anomaly detection such as UnusualCommit (\cite{Goyal2017}).
\end{itemize}
\subsection{Results for Machine Learning Approaches}
\begin{itemize}
\item Describe the results for machine learning approaches such as Deeper (\cite{Yang2015}) or CLEVER (\cite{Nayrolles2018}, could also be a metric-based approach).
\end{itemize}



\section{Discussion of Comparison Results}
\begin{itemize}
	\item Emphasize interesting findings from the results.
	\item Describe how the specified categories of Commit Validation techniques can be mapped to use cases.
	\item Outline how this mapping can be used to find an suitable Commit Validation technique for a new project.
\end{itemize}


\section{Related Surveys on Commit Validation}
\begin{itemize}
	\item Discuss and describe other surveys such as \cite{Kim2008,Catolino2019,Syed2019,Yang2016}.
	\item Highlight for each survey how its scope differs from the scope of this paper.
\end{itemize}


\section{Conclusions}

\subsection{Summary}
\begin{itemize}
	\item Give a short summary on the comparison scheme and which relevant criteria have been defined.
	\item Highlight interesting findings that were derived from the comparison results.
\end{itemize}

\subsection{Contributions}
\begin{itemize}
	\item Highlight the contributions that the proposed comparison scheme did to the scientific topic of Commit Validation.
	\item Discuss the relevance of the findings in regards to the scientific topic.
\end{itemize}

\subsection{Future Work}
\begin{itemize}
	\item Discuss relevant parts that have not been covered by this work.
	\item Highlight next steps to conduct to improve results.
\end{itemize}
