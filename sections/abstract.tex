%% LaTeX2e class for seminar theses
%% sections/abstract_en.tex
%% 
%% Karlsruhe Institute of Technology
%% Institute for Program Structures and Data Organization
%% Chair for Software Design and Quality (SDQ)
%%
%% Dr.-Ing. Erik Burger
%% burger@kit.edu
%%
%% Version 1.0, 2018-04-16

Commit Validation is an emerging topic in the area of Software Quality Assurance which can significantly reduce costs by decreasing development effort and faults by finding and fixing them when they are introduced. While many varying tools and research publications have appeared on this topic, a clear state of the art approach 
%has not been defined yet, and new methods continue to emerge. 
cannot be defined yet, and new methods continue to emerge.
The goal of this paper is to explore how Commit Validation techniques can be compared in an objective way as well as reporting on how to find a suitable Commit Validation technique for a Software Engineering project.
In an effort to satisfy this goal, five relevant Commit Validation approaches have been selected and analyzed: CLEVER, Commit Guru, Unusual Commits, Deeper and an approach by Kamei et al.
Contributions of this paper include the proposal of an objective evaluation schema for Commit Validation methods, the analysis of the five approaches and an evaluation.